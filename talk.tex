\documentclass[presentation]{beamer}
\usepackage[utf8]{inputenc}
\usepackage[T1]{fontenc}
\usepackage[ngerman]{babel}
\usepackage{graphicx}
\usepackage{textcomp}
\usepackage{marvosym}
\usetheme{Szeged}
\usecolortheme{seagull}
\setbeamertemplate{blocks}[rectangles]
\hypersetup{colorlinks=true, urlcolor=blue}

\title{Grundlagen zu verteilten Versionskontrollsystemen}
\author[Michael Markert]{Michael Markert\\
  \href{mailto:markert.michael@googlemail.com}{markert.michael@googlemail.com}\\
  \url{https://github.com/cofi}}
\date{\today}

\AtBeginSection[]
{
  \begin{frame}<beamer>
    \tableofcontents[currentsection]
  \end{frame}
}

\begin{document}
\begin{frame}[plain]
  \titlepage
\end{frame}

\begin{frame}{Outline}
  \setcounter{tocdepth}{1}
  \tableofcontents
\end{frame}
\section{Versionskontrolle}
\section{CVCS vs DVCS}
\subsection{CVCS}
\subsection{DVCS}

\section{Strategien}
\subsection{Entwicklungsstrategien}
\subsection{Branchstrategien}

\section{Subversion}
\section{Git}
\section{Mercurial}

\section{Anlaufstellen}
\subsection{Hosting}
\begin{frame}{Git}
  \begin{itemize}
  \item <1-> \href{http://github.com}{github.com}
    \begin{itemize}
    \item Wiki, Website, Issue Tracker, Teams
    \item Unbegrenzt öffentliche Repositories
    \item Für Studenten: Auf Nachfrage private Repositories
    \end{itemize}
  \item<2-> \href{http://unfuddle.com}{unfuddle.com}
    \begin{itemize}
    \item Issue Tracker
    \item 1 privates Repository
    \end{itemize}
  \item<3-> \href{http://gitorious.org}{gitorious.org}
    \begin{itemize}
    \item Wiki, Issue Tracker, Teams
    \item Unbegrenzt öffentliche Repositories
    \end{itemize}
  \item<4-> ...
  \end{itemize}
\end{frame}
\begin{frame}{Mercurial}
  \begin{itemize}
  \item<1-> \href{http://bitbucket.org}{bitbucket.org}
    \begin{itemize}
    \item Wiki, Website, Issue Tracker, Teams
    \item Unbegrenzt öffentliche Repositories
    \item 5 private Repositories
    \end{itemize}
  \end{itemize}
\end{frame}
\begin{frame}{Subversion}
  \begin{itemize}
  \item<1-> \href{http://unfuddle.com}{unfuddle.com}
    \begin{itemize}
    \item Issue Tracker
    \item 1 privates Repository
    \end{itemize}
  \item<2-> \href{http://xp-dev.com}{xp-dev.com}
    \begin{itemize}
    \item Wiki, Issue Tracker
    \item 2 private Repositories
    \end{itemize}
  \item<3-> \href{http://www.rbg.informatik.tu-darmstadt.de/onlinehilfe/freigaben/k\#svn}{Hosting bei der RBG}
  \item<4-> ...
  \end{itemize}
\end{frame}
\begin{frame}{Alle Drei}
  \begin{itemize}
  \item<1-> \href{http://sourceforge.net}{sourceforge.net}
    \begin{itemize}
    \item Wiki, Website, Issue Tracker
    \end{itemize}
  \item<2-> \href{http://code.google.com}{code.google.com}
    \begin{itemize}
    \item Wiki, Website, Issue Tracker
    \end{itemize}
  \item<3-> \href{http://assembla.com}{assembla.com}
  \item<4-> ...
  \end{itemize}
\end{frame}
\subsection{Handhabung}
\begin{frame}{Git}
  \begin{itemize}
  \item<1-> \href{http://progit.org/book/}{Progit}
  \item<2-> \href{http://www-cs-students.stanford.edu/~blynn/gitmagic/}{Git Magic}
  \item<3-> \href{http://www.kernel.org/pub/software/scm/git/docs/gittutorial.html}{Git Tutorial} 
  \item<4-> \href{http://help.github.com/}{Github Bootcamp}
  \end{itemize}
\end{frame}
\begin{frame}{Mercurial}
  \begin{itemize}
  \item<1-> \href{http://hgbook.red-bean.com/}{Mercurial - The Definitive Guide}
  \item<2-> \href{http://hginit.com/}{hg init}
  \item<3-> \href{http://mercurial.selenic.com/wiki/UnderstandingMercurial}{Understanding Mercurial}
  \end{itemize}
\end{frame}
\begin{frame}{Subversion}
  \begin{itemize}
  \item<1-> \href{http://svnbook.red-bean.com/}{Versionskontrolle mit SVN}
  \end{itemize}
\end{frame}
\end{document}
